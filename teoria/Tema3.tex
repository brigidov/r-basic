% Options for packages loaded elsewhere
\PassOptionsToPackage{unicode}{hyperref}
\PassOptionsToPackage{hyphens}{url}
%
\documentclass[
  ignorenonframetext,
]{beamer}
\usepackage{pgfpages}
\setbeamertemplate{caption}[numbered]
\setbeamertemplate{caption label separator}{: }
\setbeamercolor{caption name}{fg=normal text.fg}
\beamertemplatenavigationsymbolsempty
% Prevent slide breaks in the middle of a paragraph
\widowpenalties 1 10000
\raggedbottom
\setbeamertemplate{part page}{
  \centering
  \begin{beamercolorbox}[sep=16pt,center]{part title}
    \usebeamerfont{part title}\insertpart\par
  \end{beamercolorbox}
}
\setbeamertemplate{section page}{
  \centering
  \begin{beamercolorbox}[sep=12pt,center]{part title}
    \usebeamerfont{section title}\insertsection\par
  \end{beamercolorbox}
}
\setbeamertemplate{subsection page}{
  \centering
  \begin{beamercolorbox}[sep=8pt,center]{part title}
    \usebeamerfont{subsection title}\insertsubsection\par
  \end{beamercolorbox}
}
\AtBeginPart{
  \frame{\partpage}
}
\AtBeginSection{
  \ifbibliography
  \else
    \frame{\sectionpage}
  \fi
}
\AtBeginSubsection{
  \frame{\subsectionpage}
}
\usepackage{lmodern}
\usepackage{amssymb,amsmath}
\usepackage{ifxetex,ifluatex}
\ifnum 0\ifxetex 1\fi\ifluatex 1\fi=0 % if pdftex
  \usepackage[T1]{fontenc}
  \usepackage[utf8]{inputenc}
  \usepackage{textcomp} % provide euro and other symbols
\else % if luatex or xetex
  \usepackage{unicode-math}
  \defaultfontfeatures{Scale=MatchLowercase}
  \defaultfontfeatures[\rmfamily]{Ligatures=TeX,Scale=1}
\fi
% Use upquote if available, for straight quotes in verbatim environments
\IfFileExists{upquote.sty}{\usepackage{upquote}}{}
\IfFileExists{microtype.sty}{% use microtype if available
  \usepackage[]{microtype}
  \UseMicrotypeSet[protrusion]{basicmath} % disable protrusion for tt fonts
}{}
\makeatletter
\@ifundefined{KOMAClassName}{% if non-KOMA class
  \IfFileExists{parskip.sty}{%
    \usepackage{parskip}
  }{% else
    \setlength{\parindent}{0pt}
    \setlength{\parskip}{6pt plus 2pt minus 1pt}}
}{% if KOMA class
  \KOMAoptions{parskip=half}}
\makeatother
\usepackage{xcolor}
\IfFileExists{xurl.sty}{\usepackage{xurl}}{} % add URL line breaks if available
\IfFileExists{bookmark.sty}{\usepackage{bookmark}}{\usepackage{hyperref}}
\hypersetup{
  pdftitle={Tema 3 - Estructuras de datos},
  pdfauthor={Juan Gabriel Gomila \& María Santos},
  hidelinks,
  pdfcreator={LaTeX via pandoc}}
\urlstyle{same} % disable monospaced font for URLs
\newif\ifbibliography
\usepackage{color}
\usepackage{fancyvrb}
\newcommand{\VerbBar}{|}
\newcommand{\VERB}{\Verb[commandchars=\\\{\}]}
\DefineVerbatimEnvironment{Highlighting}{Verbatim}{commandchars=\\\{\}}
% Add ',fontsize=\small' for more characters per line
\usepackage{framed}
\definecolor{shadecolor}{RGB}{248,248,248}
\newenvironment{Shaded}{\begin{snugshade}}{\end{snugshade}}
\newcommand{\AlertTok}[1]{\textcolor[rgb]{0.94,0.16,0.16}{#1}}
\newcommand{\AnnotationTok}[1]{\textcolor[rgb]{0.56,0.35,0.01}{\textbf{\textit{#1}}}}
\newcommand{\AttributeTok}[1]{\textcolor[rgb]{0.77,0.63,0.00}{#1}}
\newcommand{\BaseNTok}[1]{\textcolor[rgb]{0.00,0.00,0.81}{#1}}
\newcommand{\BuiltInTok}[1]{#1}
\newcommand{\CharTok}[1]{\textcolor[rgb]{0.31,0.60,0.02}{#1}}
\newcommand{\CommentTok}[1]{\textcolor[rgb]{0.56,0.35,0.01}{\textit{#1}}}
\newcommand{\CommentVarTok}[1]{\textcolor[rgb]{0.56,0.35,0.01}{\textbf{\textit{#1}}}}
\newcommand{\ConstantTok}[1]{\textcolor[rgb]{0.00,0.00,0.00}{#1}}
\newcommand{\ControlFlowTok}[1]{\textcolor[rgb]{0.13,0.29,0.53}{\textbf{#1}}}
\newcommand{\DataTypeTok}[1]{\textcolor[rgb]{0.13,0.29,0.53}{#1}}
\newcommand{\DecValTok}[1]{\textcolor[rgb]{0.00,0.00,0.81}{#1}}
\newcommand{\DocumentationTok}[1]{\textcolor[rgb]{0.56,0.35,0.01}{\textbf{\textit{#1}}}}
\newcommand{\ErrorTok}[1]{\textcolor[rgb]{0.64,0.00,0.00}{\textbf{#1}}}
\newcommand{\ExtensionTok}[1]{#1}
\newcommand{\FloatTok}[1]{\textcolor[rgb]{0.00,0.00,0.81}{#1}}
\newcommand{\FunctionTok}[1]{\textcolor[rgb]{0.00,0.00,0.00}{#1}}
\newcommand{\ImportTok}[1]{#1}
\newcommand{\InformationTok}[1]{\textcolor[rgb]{0.56,0.35,0.01}{\textbf{\textit{#1}}}}
\newcommand{\KeywordTok}[1]{\textcolor[rgb]{0.13,0.29,0.53}{\textbf{#1}}}
\newcommand{\NormalTok}[1]{#1}
\newcommand{\OperatorTok}[1]{\textcolor[rgb]{0.81,0.36,0.00}{\textbf{#1}}}
\newcommand{\OtherTok}[1]{\textcolor[rgb]{0.56,0.35,0.01}{#1}}
\newcommand{\PreprocessorTok}[1]{\textcolor[rgb]{0.56,0.35,0.01}{\textit{#1}}}
\newcommand{\RegionMarkerTok}[1]{#1}
\newcommand{\SpecialCharTok}[1]{\textcolor[rgb]{0.00,0.00,0.00}{#1}}
\newcommand{\SpecialStringTok}[1]{\textcolor[rgb]{0.31,0.60,0.02}{#1}}
\newcommand{\StringTok}[1]{\textcolor[rgb]{0.31,0.60,0.02}{#1}}
\newcommand{\VariableTok}[1]{\textcolor[rgb]{0.00,0.00,0.00}{#1}}
\newcommand{\VerbatimStringTok}[1]{\textcolor[rgb]{0.31,0.60,0.02}{#1}}
\newcommand{\WarningTok}[1]{\textcolor[rgb]{0.56,0.35,0.01}{\textbf{\textit{#1}}}}
\usepackage{graphicx,grffile}
\makeatletter
\def\maxwidth{\ifdim\Gin@nat@width>\linewidth\linewidth\else\Gin@nat@width\fi}
\def\maxheight{\ifdim\Gin@nat@height>\textheight\textheight\else\Gin@nat@height\fi}
\makeatother
% Scale images if necessary, so that they will not overflow the page
% margins by default, and it is still possible to overwrite the defaults
% using explicit options in \includegraphics[width, height, ...]{}
\setkeys{Gin}{width=\maxwidth,height=\maxheight,keepaspectratio}
% Set default figure placement to htbp
\makeatletter
\def\fps@figure{htbp}
\makeatother
\setlength{\emergencystretch}{3em} % prevent overfull lines
\providecommand{\tightlist}{%
  \setlength{\itemsep}{0pt}\setlength{\parskip}{0pt}}
\setcounter{secnumdepth}{-\maxdimen} % remove section numbering

\title{Tema 3 - Estructuras de datos}
\author{Juan Gabriel Gomila \& María Santos}
\date{}

\begin{document}
\frame{\titlepage}

\hypertarget{vectores}{%
\section{Vectores}\label{vectores}}

\begin{frame}[fragile]{Tipos de datos en R}
\protect\hypertarget{tipos-de-datos-en-r}{}

Un \textbf{vector} es una secuencia ordenada de datos. \texttt{R}
dispone de muchos tipos de datos, por ejemplo:

\begin{itemize}
\tightlist
\item
  \texttt{logical}: lógicos (\texttt{TRUE} o \texttt{FALSE})
\item
  \texttt{integer}: números enteros, \(\mathbb Z\)
\item
  \texttt{numeric}: números reales, \(\mathbb R\)
\item
  \texttt{complex}: números complejos, \(\mathbb C\)
\item
  \texttt{character}: palabras
\end{itemize}

En los vectores de \texttt{R}, todos sus objetos han de ser del mismo
tipo: todos números, todos palabras, etc. Cuando queramos usar vectores
formados por objetos de diferentes tipos, tendremos que usar
\textbf{listas generalizadas}, \texttt{lists} que veremos al final del
tema.

\end{frame}

\begin{frame}[fragile]{Básico}
\protect\hypertarget{buxe1sico}{}

\begin{itemize}
\tightlist
\item
  \texttt{c()}: para definir un vector
\item
  \texttt{scan()}: para definir un vector
\item
  \texttt{fix(x)}: para modificar visualmente el vector \(x\)
\item
  \texttt{rep(a,n)}: para definir un vector constante que contiene el
  dato \(a\) repetido \(n\) veces
\end{itemize}

\begin{Shaded}
\begin{Highlighting}[]
\KeywordTok{c}\NormalTok{(}\DecValTok{1}\NormalTok{,}\DecValTok{2}\NormalTok{,}\DecValTok{3}\NormalTok{)}
\end{Highlighting}
\end{Shaded}

\begin{verbatim}
[1] 1 2 3
\end{verbatim}

\begin{Shaded}
\begin{Highlighting}[]
\KeywordTok{rep}\NormalTok{(}\StringTok{"Mates"}\NormalTok{,}\DecValTok{7}\NormalTok{)}
\end{Highlighting}
\end{Shaded}

\begin{verbatim}
[1] "Mates" "Mates" "Mates" "Mates" "Mates" "Mates" "Mates"
\end{verbatim}

\end{frame}

\begin{frame}{Función scan()}
\protect\hypertarget{funciuxf3n-scan}{}

\textbf{Ejemplo}

Vamos a crear un vector que contenga 3 copias de 1 9 9 8 0 7 2 6 con la
función scan:

\includegraphics{Imgs/scan.png}

\end{frame}

\begin{frame}{Básico}
\protect\hypertarget{buxe1sico-1}{}

\textbf{Ejercicio}

\begin{enumerate}
\tightlist
\item
  Repite tu año de nacimiento 10 veces 
\item
  Crea el vector que tenga como entradas
  \(16, 0, 1, 20, 1, 7, 88, 5, 1, 9\), llámalo vec y modifica la cuarta
  entrada con la función fix() 
\end{enumerate}

\includegraphics{Imgs/studytime.png}

\end{frame}

\begin{frame}[fragile]{Progresiones y Secuencias}
\protect\hypertarget{progresiones-y-secuencias}{}

Una progresión aritmética es una sucesión de números tales que la
\textbf{diferencia}, \(d\), de cualquier par de términos sucesivos de la
secuencia es constante. \[a_n = a_1 + (n-1)\cdot d\]

\begin{itemize}
\tightlist
\item
  \texttt{seq(a,b,by=d)}: para generar una
  \href{https://es.wikipedia.org/wiki/Progresión_aritmética}{progresión
  aritmética} de diferencia \(d\) que empieza en \(a\) hasta llegar a
  \(b\)
\item
  \texttt{seq(a,b,\ length.out=n)}: define progresión aritmética de
  longitud \(n\) que va de \(a\) a \(b\) con diferencia \(d\). Por tanto
  \(d=(b-a)/(n-1)\)
\item
  \texttt{seq(a,by=d,\ length.out=n)}: define la progresión aritmética
  de longitud \(n\) y diferencia \(d\) que empieza en \(a\)
\item
  \texttt{a:b}: define la secuencia de números \textbf{enteros}
  (\(\mathbb{Z}\)) consecutivos entre dos números \(a\) y \(b\)
\end{itemize}

\end{frame}

\begin{frame}{Secuencias}
\protect\hypertarget{secuencias}{}

\textbf{Ejercicio}

\begin{itemize}
\tightlist
\item
  Imprimid los números del 1 al 20 
\item
  Imprimid los 20 primeros números pares 
\item
  Imprimid 30 números equidistantes entre el 17 y el 98, mostrando solo
  4 cifras significativas 
\end{itemize}

\includegraphics{Imgs/studytime.png}

\end{frame}

\begin{frame}[fragile]{Funciones}
\protect\hypertarget{funciones}{}

Cuando queremos aplicar una función a cada uno de los elementos de un
vector de datos, la función \texttt{sapply} nos ahorra tener que
programar con bucles en \texttt{R}:

\begin{itemize}
\tightlist
\item
  \texttt{sapply(nombre\_de\_vector,FUN=nombre\_de\_función)}: para
  aplicar dicha función a todos los elementos del vector
\item
  \texttt{sqrt(x)}: calcula un nuevo vector con las raíces cuadradas de
  cada uno de los elementos del vector \(x\)
\end{itemize}

\end{frame}

\begin{frame}[fragile]{Funciones}
\protect\hypertarget{funciones-1}{}

Dado un vector de datos \(x\) podemos calcular muchas medidas
estadísticas acerca del mismo:

\begin{itemize}
\tightlist
\item
  \texttt{length(x)}: calcula la longitud del vector \(x\)
\item
  \texttt{max(x)}: calcula el máximo del vector \(x\)
\item
  \texttt{min(x)}: calcula el mínimo del vector \(x\)
\item
  \texttt{sum(x)}: calcula la suma de las entradas del vector \(x\)
\item
  \texttt{prod(x)}: calcula el producto de las entradas del vector \(x\)
\end{itemize}

\end{frame}

\begin{frame}[fragile]{Funciones}
\protect\hypertarget{funciones-2}{}

\begin{itemize}
\tightlist
\item
  \texttt{mean(x)}: calcula la media aritmética de las entradas del
  vector \(x\)
\item
  \texttt{diff(x)}: calcula el vector formado por las diferencias
  sucesivas entre entradas del vector original \(x\)
\item
  \texttt{cumsum(x)}: calcula el vector formado por las sumas acumuladas
  de las entradas del vector original \(x\)

  \begin{itemize}
  \tightlist
  \item
    Permite definir sucesiones descritas mediante sumatorios
  \item
    Cada entrada de \texttt{cumsum(x)} es la suma de las entradas de
    \(x\) hasta su posición
  \end{itemize}
\end{itemize}

\end{frame}

\begin{frame}[fragile]{Funciones}
\protect\hypertarget{funciones-3}{}

\begin{Shaded}
\begin{Highlighting}[]
\NormalTok{cuadrado =}\StringTok{ }\ControlFlowTok{function}\NormalTok{(x)\{x}\OperatorTok{^}\DecValTok{2}\NormalTok{\}}
\NormalTok{v =}\StringTok{ }\KeywordTok{c}\NormalTok{(}\DecValTok{1}\NormalTok{,}\DecValTok{2}\NormalTok{,}\DecValTok{3}\NormalTok{,}\DecValTok{4}\NormalTok{,}\DecValTok{5}\NormalTok{,}\DecValTok{6}\NormalTok{)}
\KeywordTok{sapply}\NormalTok{(v, }\DataTypeTok{FUN =}\NormalTok{ cuadrado)}
\end{Highlighting}
\end{Shaded}

\begin{verbatim}
[1]  1  4  9 16 25 36
\end{verbatim}

\begin{Shaded}
\begin{Highlighting}[]
\KeywordTok{mean}\NormalTok{(v)}
\end{Highlighting}
\end{Shaded}

\begin{verbatim}
[1] 3.5
\end{verbatim}

\begin{Shaded}
\begin{Highlighting}[]
\KeywordTok{cumsum}\NormalTok{(v)}
\end{Highlighting}
\end{Shaded}

\begin{verbatim}
[1]  1  3  6 10 15 21
\end{verbatim}

\end{frame}

\begin{frame}[fragile]{Orden}
\protect\hypertarget{orden}{}

\begin{itemize}
\tightlist
\item
  \texttt{sort(x)}: ordena el vector en orden natural de los objetos que
  lo forman: el orden numérico creciente, orden alfabético\ldots{}
\item
  \texttt{rev(x)}: invierte el orden de los elementos del vector \(x\)
\end{itemize}

\begin{Shaded}
\begin{Highlighting}[]
\NormalTok{v =}\StringTok{ }\KeywordTok{c}\NormalTok{(}\DecValTok{1}\NormalTok{,}\DecValTok{7}\NormalTok{,}\DecValTok{5}\NormalTok{,}\DecValTok{2}\NormalTok{,}\DecValTok{4}\NormalTok{,}\DecValTok{6}\NormalTok{,}\DecValTok{3}\NormalTok{)}
\KeywordTok{sort}\NormalTok{(v)}
\end{Highlighting}
\end{Shaded}

\begin{verbatim}
[1] 1 2 3 4 5 6 7
\end{verbatim}

\begin{Shaded}
\begin{Highlighting}[]
\KeywordTok{rev}\NormalTok{(v)}
\end{Highlighting}
\end{Shaded}

\begin{verbatim}
[1] 3 6 4 2 5 7 1
\end{verbatim}

\end{frame}

\begin{frame}[fragile]{Orden}
\protect\hypertarget{orden-1}{}

\textbf{Ejercicio}

\begin{itemize}
\item
  Combinad las dos funciones anteriores, \texttt{sort} y \texttt{rev}
  para crear una función que dado un vector \(x\) os lo devuelva
  ordenado en orden decreciente.
\item
  Razonad si aplicar primero \texttt{sort} y luego \texttt{rev} a un
  vector \(x\) daría en general el mismo resultado que aplicar primero
  \texttt{rev} y luego \texttt{sort}.
\item
  Investigad la documentación de la función \texttt{sort} (recordad que
  podéis usar la sintaxis \texttt{?sort} en la consola) para leer si
  cambiando algún argumento de la misma podéis obtener el mismo
  resultado que habéis programado en el primer ejercicio.
\end{itemize}

\end{frame}

\begin{frame}[fragile]{Subvectores}
\protect\hypertarget{subvectores}{}

\begin{itemize}
\item
  \texttt{vector{[}i{]}}: da la \(i\)-ésima entrada del vector

  \begin{itemize}
  \tightlist
  \item
    Los índices en R empiezan en 1
  \item
    \texttt{vector{[}length(vector){]}}: nos da la última entrada del
    vector
  \item
    \texttt{vector{[}a:b{]}}: si \(a\) y \(b\) son dos números
    naturales, nos da el subvector con las entradas del vector original
    que van de la posición \(a\)-ésima hasta la \(b\)-ésima.
  \item
    \texttt{vector{[}-i{]}}: si \(i\) es un número, este subvector está
    formado por todas las entradas del vector original menos la entrada
    \(i\)-ésima. Si \(i\) resulta ser un vector, entonces es un vector
    de índices y crea un nuevo vector con las entradas del vector
    original,cuyos índices pertenecen a \(i\)
  \item
    \texttt{vector{[}-x{]}}: si \(x\) es un vector (de índices),
    entonces este es el complementario de vector{[}\(x\){]}
  \end{itemize}
\end{itemize}

\end{frame}

\begin{frame}[fragile]{Subvectores}
\protect\hypertarget{subvectores-1}{}

\begin{itemize}
\item
  También podemos utilizar operadores lógicos:

  \begin{itemize}
  \tightlist
  \item
    \texttt{==}: =
  \item
    \texttt{!=}: \(\neq\)
  \item
    \texttt{\textgreater{}=}: \(\ge\)\\
  \item
    \texttt{\textless{}=}: \(\le\)
  \item
    \texttt{\textless{}}: \(<\)
  \item
    \texttt{\textgreater{}}: \(>\)
  \item
    \texttt{!}: NO lógico
  \item
    \texttt{\&}: Y lógico
  \item
    \texttt{\textbar{}}: O lógico
  \end{itemize}
\end{itemize}

\end{frame}

\begin{frame}[fragile]{Subvectores}
\protect\hypertarget{subvectores-2}{}

\begin{Shaded}
\begin{Highlighting}[]
\NormalTok{v =}\StringTok{ }\KeywordTok{c}\NormalTok{(}\DecValTok{14}\NormalTok{,}\DecValTok{5}\NormalTok{,}\DecValTok{6}\NormalTok{,}\DecValTok{19}\NormalTok{,}\DecValTok{32}\NormalTok{,}\DecValTok{0}\NormalTok{,}\DecValTok{8}\NormalTok{)}
\NormalTok{v[}\DecValTok{2}\NormalTok{]}
\end{Highlighting}
\end{Shaded}

\begin{verbatim}
[1] 5
\end{verbatim}

\begin{Shaded}
\begin{Highlighting}[]
\NormalTok{v[}\OperatorTok{-}\KeywordTok{c}\NormalTok{(}\DecValTok{3}\NormalTok{,}\DecValTok{5}\NormalTok{)]}
\end{Highlighting}
\end{Shaded}

\begin{verbatim}
[1] 14  5 19  0  8
\end{verbatim}

\begin{Shaded}
\begin{Highlighting}[]
\NormalTok{v[v }\OperatorTok{!=}\StringTok{ }\DecValTok{19} \OperatorTok{&}\StringTok{ }\NormalTok{v}\OperatorTok{>}\DecValTok{15}\NormalTok{]}
\end{Highlighting}
\end{Shaded}

\begin{verbatim}
[1] 32
\end{verbatim}

\end{frame}

\begin{frame}[fragile]{Condicionales}
\protect\hypertarget{condicionales}{}

\begin{itemize}
\tightlist
\item
  \texttt{which(x\ cumple\ condición)}: para obtener los índices de las
  entradas del vector \(x\) que satisfacen la condición dada
\item
  \texttt{which.min(x)}: nos da la primera posición en la que el vector
  \(x\) toma su valor mínimo
\item
  \texttt{which(x==min(x))}: da todas las posiciones en las que el
  vector \(x\) toma sus valores mínimos
\item
  \texttt{which.max(x)}: nos da la primera posición en la que el vector
  \(x\) toma su valor máximo
\item
  \texttt{which(x==max(x))}: da todas las posiciones en las que el
  vector \(x\) toma sus valores máximos
\end{itemize}

\end{frame}

\hypertarget{factores}{%
\section{Factores}\label{factores}}

\begin{frame}[fragile]{Factor}
\protect\hypertarget{factor}{}

Factor: es como un vector, pero con una estructura interna más rica que
permite usarlo para clasificar observaciones

\begin{itemize}
\tightlist
\item
  \texttt{levels}: atributo del factor. Cada elemento del factor es
  igual a un nivel. Los niveles clasifican las entradas del factor. Se
  ordenan por orden alfabético
\item
  Para definir un factor, primero hemos de definir un vector y
  trasformarlo por medio de una de las funciones \texttt{factor()} o
  \texttt{as.factor()}.
\end{itemize}

\end{frame}

\begin{frame}[fragile]{La función factor()}
\protect\hypertarget{la-funciuxf3n-factor}{}

\begin{itemize}
\item
  \texttt{factor(vector,levels=...)}: define un factor a partir del
  vector y dispone de algunos parámetros que permiten modificar el
  factor que se crea:

  \begin{itemize}
  \tightlist
  \item
    \texttt{levels}: permite especificar los niveles e incluso añadir
    niveles que no aparecen en el vector
  \item
    \texttt{labels}: permite cambiar los nombres de los niveles
  \end{itemize}
\item
  \texttt{levels(factor)}: para obtener los niveles del factor
\end{itemize}

\end{frame}

\begin{frame}[fragile]{Factor ordenado}
\protect\hypertarget{factor-ordenado}{}

Factor ordenado. Es un factor donde los niveles siguen un orden

\begin{itemize}
\tightlist
\item
  \texttt{ordered(vector,levels=...)}: función que define un factor
  ordenado y tiene los mismos parámetros que factor
\end{itemize}

\end{frame}

\begin{frame}[fragile]{Factores y factores ordenados}
\protect\hypertarget{factores-y-factores-ordenados}{}

\begin{Shaded}
\begin{Highlighting}[]
\NormalTok{fac =}\StringTok{ }\KeywordTok{factor}\NormalTok{(}\KeywordTok{c}\NormalTok{(}\DecValTok{1}\NormalTok{,}\DecValTok{1}\NormalTok{,}\DecValTok{1}\NormalTok{,}\DecValTok{2}\NormalTok{,}\DecValTok{2}\NormalTok{,}\DecValTok{3}\NormalTok{,}\DecValTok{2}\NormalTok{,}\DecValTok{4}\NormalTok{,}\DecValTok{1}\NormalTok{,}\DecValTok{3}\NormalTok{,}\DecValTok{3}\NormalTok{,}\DecValTok{4}\NormalTok{,}\DecValTok{2}\NormalTok{,}\DecValTok{3}\NormalTok{,}\DecValTok{4}\NormalTok{,}\DecValTok{4}\NormalTok{), }
       \DataTypeTok{levels =} \KeywordTok{c}\NormalTok{(}\DecValTok{1}\NormalTok{,}\DecValTok{2}\NormalTok{,}\DecValTok{3}\NormalTok{,}\DecValTok{4}\NormalTok{), }\DataTypeTok{labels =} \KeywordTok{c}\NormalTok{(}\StringTok{"Sus"}\NormalTok{,}\StringTok{"Apr"}\NormalTok{,}\StringTok{"Not"}\NormalTok{,}\StringTok{"Exc"}\NormalTok{))}
\NormalTok{fac}
\end{Highlighting}
\end{Shaded}

\begin{verbatim}
 [1] Sus Sus Sus Apr Apr Not Apr Exc Sus Not Not Exc Apr Not Exc Exc
Levels: Sus Apr Not Exc
\end{verbatim}

\begin{Shaded}
\begin{Highlighting}[]
\NormalTok{facOrd =}\StringTok{ }\KeywordTok{ordered}\NormalTok{(}\KeywordTok{c}\NormalTok{(}\DecValTok{1}\NormalTok{,}\DecValTok{1}\NormalTok{,}\DecValTok{1}\NormalTok{,}\DecValTok{2}\NormalTok{,}\DecValTok{2}\NormalTok{,}\DecValTok{3}\NormalTok{,}\DecValTok{2}\NormalTok{,}\DecValTok{4}\NormalTok{,}\DecValTok{1}\NormalTok{,}\DecValTok{3}\NormalTok{,}\DecValTok{3}\NormalTok{,}\DecValTok{4}\NormalTok{,}\DecValTok{2}\NormalTok{,}\DecValTok{3}\NormalTok{,}\DecValTok{4}\NormalTok{,}\DecValTok{4}\NormalTok{), }
       \DataTypeTok{levels =} \KeywordTok{c}\NormalTok{(}\DecValTok{1}\NormalTok{,}\DecValTok{2}\NormalTok{,}\DecValTok{3}\NormalTok{,}\DecValTok{4}\NormalTok{), }\DataTypeTok{labels =} \KeywordTok{c}\NormalTok{(}\StringTok{"Sus"}\NormalTok{,}\StringTok{"Apr"}\NormalTok{,}\StringTok{"Not"}\NormalTok{,}\StringTok{"Exc"}\NormalTok{))}
\NormalTok{facOrd}
\end{Highlighting}
\end{Shaded}

\begin{verbatim}
 [1] Sus Sus Sus Apr Apr Not Apr Exc Sus Not Not Exc Apr Not Exc Exc
Levels: Sus < Apr < Not < Exc
\end{verbatim}

\end{frame}

\hypertarget{lists}{%
\section{Lists}\label{lists}}

\begin{frame}[fragile]{List}
\protect\hypertarget{list}{}

List. Lista formada por diferentes objetos, no necesariamente del mismo
tipo, cada cual con un nombre interno

\begin{itemize}
\tightlist
\item
  \texttt{list(...)}: función que crea una list

  \begin{itemize}
  \tightlist
  \item
    Para obtener una componente concreta usamos la instrucción
    \texttt{list\$componente}
  \item
    También podemos indicar el objeto por su posición usando dobles
    corchetes: \texttt{list{[}{[}i{]}{]}}. Lo que obtendremos es una
    list formada por esa única componente, no el objeto que forma la
    componente
  \end{itemize}
\end{itemize}

\end{frame}

\begin{frame}[fragile]{Obtener información de una list}
\protect\hypertarget{obtener-informaciuxf3n-de-una-list}{}

\begin{itemize}
\tightlist
\item
  \texttt{str(list)}: para conocer la estructura interna de una list
\item
  \texttt{names(list)}: para saber los nombres de la list
\end{itemize}

\end{frame}

\begin{frame}[fragile]{Obtener información de una list}
\protect\hypertarget{obtener-informaciuxf3n-de-una-list-1}{}

\begin{Shaded}
\begin{Highlighting}[]
\NormalTok{x =}\StringTok{ }\KeywordTok{c}\NormalTok{(}\DecValTok{1}\NormalTok{,}\OperatorTok{-}\DecValTok{2}\NormalTok{,}\DecValTok{3}\NormalTok{,}\DecValTok{4}\NormalTok{,}\OperatorTok{-}\DecValTok{5}\NormalTok{,}\DecValTok{6}\NormalTok{,}\DecValTok{7}\NormalTok{,}\OperatorTok{-}\DecValTok{8}\NormalTok{,}\OperatorTok{-}\DecValTok{9}\NormalTok{,}\DecValTok{0}\NormalTok{)}
\NormalTok{miLista =}\StringTok{ }\KeywordTok{list}\NormalTok{(}\DataTypeTok{nombre =} \StringTok{"X"}\NormalTok{, }\DataTypeTok{vector =}\NormalTok{ x, }\DataTypeTok{media =} \KeywordTok{mean}\NormalTok{(x), }\DataTypeTok{sumas =} \KeywordTok{cumsum}\NormalTok{(x))}
\NormalTok{miLista}
\end{Highlighting}
\end{Shaded}

\begin{verbatim}
$nombre
[1] "X"

$vector
 [1]  1 -2  3  4 -5  6  7 -8 -9  0

$media
[1] -0.3

$sumas
 [1]  1 -1  2  6  1  7 14  6 -3 -3
\end{verbatim}

\end{frame}

\begin{frame}[fragile]{Obtener información de una list}
\protect\hypertarget{obtener-informaciuxf3n-de-una-list-2}{}

\begin{Shaded}
\begin{Highlighting}[]
\KeywordTok{str}\NormalTok{(miLista)}
\end{Highlighting}
\end{Shaded}

\begin{verbatim}
List of 4
 $ nombre: chr "X"
 $ vector: num [1:10] 1 -2 3 4 -5 6 7 -8 -9 0
 $ media : num -0.3
 $ sumas : num [1:10] 1 -1 2 6 1 7 14 6 -3 -3
\end{verbatim}

\begin{Shaded}
\begin{Highlighting}[]
\KeywordTok{names}\NormalTok{(miLista)}
\end{Highlighting}
\end{Shaded}

\begin{verbatim}
[1] "nombre" "vector" "media"  "sumas" 
\end{verbatim}

\end{frame}

\hypertarget{matrices}{%
\section{Matrices}\label{matrices}}

\begin{frame}[fragile]{Cómo definirlas}
\protect\hypertarget{cuxf3mo-definirlas}{}

\begin{itemize}
\tightlist
\item
  \texttt{matrix(vector,\ nrow=n,\ byrow=valor\_lógico)}: para definir
  una matriz de \(n\) filas formada por las entradas del vector

  \begin{itemize}
  \tightlist
  \item
    \texttt{nrow}: número de filas
  \item
    \texttt{byrow}: si se iguala a TRUE, la matriz se construye por
    filas; si se iguala a FALSE (valor por defecto), se construye por
    columnas. -\texttt{ncol}: número de columnas (puede usarse en lugar
    de nrow)
  \item
    R muestra las matrices indicando como {[}\(i,\){]} la fila
    \(i\)-ésima y {[}\(,j\){]} la columna \(j\)-ésima
  \item
    Todas las entradas de una matriz han de ser del mismo tipo de datos
  \end{itemize}
\end{itemize}

\end{frame}

\begin{frame}{Cómo definirlas}
\protect\hypertarget{cuxf3mo-definirlas-1}{}

\textbf{Ejercicio}

\begin{itemize}
\item
  ¿Cómo definirías una matriz constante? Es decir, ¿cómo definirías una
  matriz \(A\) tal que \(\forall\  i=1,...,n; j = 1,...,m\),
  \(a_{i,j}=k\) siendo \(k\in\mathbb{R}\)? Como R no admite incógnitas,
  prueba para el caso específico \(n = 3, m = 5, k = 0\) 
\item
  Con el vector vec = (1,2,3,4,5,6,7,8,9,10,11,12) crea la matriz
  \[\begin{pmatrix}
  1 & 4 & 7 & 10\\
  2 & 5 & 8 & 11\\
  3 & 6 & 9 & 12
  \end{pmatrix}\] 
\end{itemize}

\end{frame}

\begin{frame}[fragile]{Cómo construirlas}
\protect\hypertarget{cuxf3mo-construirlas}{}

\begin{itemize}
\tightlist
\item
  \texttt{rbind(vector1,\ vector2,\ ...)}: construye la matriz de filas
  vector1, vector2,\ldots{}
\item
  \texttt{cbind(vector1,\ vector2,\ ...)}: construye la matriz de
  columnas vector1, vector2,\ldots{}

  \begin{itemize}
  \tightlist
  \item
    Los vectores han de tener la misma longitud
  \item
    También sirve para añadir columnas (filas) a una matriz o concatenar
    por columnas (filas) matrices con el mismo número de filas
    (columnas)
  \end{itemize}
\item
  \texttt{diag(vector)}: para construir una matriz diagonal con un
  vector dado

  \begin{itemize}
  \tightlist
  \item
    Si aplicamos diag a un número \(n\), produce una matriz identidad de
    orden \(n\)
  \end{itemize}
\end{itemize}

\end{frame}

\begin{frame}[fragile]{Submatrices}
\protect\hypertarget{submatrices}{}

\begin{itemize}
\tightlist
\item
  \texttt{matriz{[}i,j{]}}: indica la entrada (\(i,j\)) de la matriz,
  siendo \(i,j\in\mathbb{N}\). Si \(i\) y \(j\) son vectores de índices,
  estaremos definiendo la submatriz con las filas pertenecientes al
  vector \(i\) y columnas pertenecientes al vector \(j\)
\item
  \texttt{matriz{[}i,{]}}: indica la fila \(i\)-ésima de la matriz,
  siendo \(i\in\mathbb{N}\)
\item
  \texttt{matriz{[},j{]}}: indica la columna \(j\)-ésima de la siendo
  \(j\in\mathbb{N}\)

  \begin{itemize}
  \tightlist
  \item
    Si \(i\) (\(j\)) es un vector de índices, estaremos definiendo la
    submatriz con las filas (columnas) pertenecientes al vector \(i\)
    (\(j\))
  \end{itemize}
\end{itemize}

\end{frame}

\begin{frame}[fragile]{Funciones}
\protect\hypertarget{funciones-4}{}

\begin{itemize}
\tightlist
\item
  \texttt{diag(matriz)}: para obtener la diagonal de la matriz
\item
  \texttt{nrow(matriz)}: nos devuelve el número de filas de la matriz
\item
  \texttt{ncol(matriz)}: nos devuelve el número de columnas de la matriz
\item
  \texttt{dim(matriz)}: nos devuelve las dimensiones de la matriz
\item
  \texttt{sum(matriz)}: obtenemos la suma de todas las entradas de la
  matriz
\item
  \texttt{prod(matriz)}: obtenemos el producto de todas las entradas de
  la matriz
\item
  \texttt{mean(matriz)}: obtenemos la media aritmética de todas las
  entradas de la matriz
\end{itemize}

\end{frame}

\begin{frame}[fragile]{Funciones}
\protect\hypertarget{funciones-5}{}

\begin{itemize}
\tightlist
\item
  \texttt{colSums(matriz)}: obtenemos las sumas por columnas de la
  matriz
\item
  \texttt{rowSums(matriz)}: obtenemos las sumas por filas de la matriz
\item
  \texttt{colMeans(matriz)}: obtenemos las medias aritméticas por
  columnas de la matriz
\item
  \texttt{rowMeans(matriz)}: obtenemos las medias aritméticas por filas
  de la matriz
\end{itemize}

\end{frame}

\begin{frame}[fragile]{Funciones}
\protect\hypertarget{funciones-6}{}

\textbf{Ejemplo}

Dada la matriz \[A = \begin{pmatrix}
1 & 4 & 7\\
2 & 5 & 8\\
3 & 6 & 9
\end{pmatrix}\]

\begin{Shaded}
\begin{Highlighting}[]
\NormalTok{A =}\StringTok{ }\KeywordTok{matrix}\NormalTok{(}\KeywordTok{c}\NormalTok{(}\DecValTok{1}\NormalTok{,}\DecValTok{2}\NormalTok{,}\DecValTok{3}\NormalTok{,}\DecValTok{4}\NormalTok{,}\DecValTok{5}\NormalTok{,}\DecValTok{6}\NormalTok{,}\DecValTok{7}\NormalTok{,}\DecValTok{8}\NormalTok{,}\DecValTok{9}\NormalTok{), }\DataTypeTok{ncol =} \DecValTok{3}\NormalTok{)}
\KeywordTok{dim}\NormalTok{(A)}
\end{Highlighting}
\end{Shaded}

\begin{verbatim}
[1] 3 3
\end{verbatim}

\begin{Shaded}
\begin{Highlighting}[]
\KeywordTok{diag}\NormalTok{(A)}
\end{Highlighting}
\end{Shaded}

\begin{verbatim}
[1] 1 5 9
\end{verbatim}

\end{frame}

\begin{frame}[fragile]{Función apply()}
\protect\hypertarget{funciuxf3n-apply}{}

\begin{itemize}
\tightlist
\item
  \texttt{apply(matriz,\ MARGIN=...,\ FUN=función)}: para aplicar otras
  funciones a las filas o las columnas de una matriz

  \begin{itemize}
  \tightlist
  \item
    \texttt{MARGIN}: ha de ser 1 si queremos aplicar la función por
    filas; 2 si queremos aplicarla por columnas; o c(1,2) si la queremos
    aplicar a cada entrada
  \end{itemize}
\end{itemize}

\end{frame}

\begin{frame}[fragile]{Función apply()}
\protect\hypertarget{funciuxf3n-apply-1}{}

\begin{Shaded}
\begin{Highlighting}[]
\KeywordTok{apply}\NormalTok{(A, }\DataTypeTok{MARGIN =} \KeywordTok{c}\NormalTok{(}\DecValTok{1}\NormalTok{,}\DecValTok{2}\NormalTok{), }\DataTypeTok{FUN =}\NormalTok{ cuadrado)}
\end{Highlighting}
\end{Shaded}

\begin{verbatim}
     [,1] [,2] [,3]
[1,]    1   16   49
[2,]    4   25   64
[3,]    9   36   81
\end{verbatim}

\begin{Shaded}
\begin{Highlighting}[]
\KeywordTok{apply}\NormalTok{(A, }\DataTypeTok{MARGIN =} \DecValTok{1}\NormalTok{, }\DataTypeTok{FUN =}\NormalTok{ sum)}
\end{Highlighting}
\end{Shaded}

\begin{verbatim}
[1] 12 15 18
\end{verbatim}

\begin{Shaded}
\begin{Highlighting}[]
\KeywordTok{apply}\NormalTok{(A, }\DataTypeTok{MARGIN =} \DecValTok{2}\NormalTok{, }\DataTypeTok{FUN =}\NormalTok{ sum)}
\end{Highlighting}
\end{Shaded}

\begin{verbatim}
[1]  6 15 24
\end{verbatim}

\end{frame}

\begin{frame}[fragile]{Operaciones}
\protect\hypertarget{operaciones}{}

\begin{itemize}
\tightlist
\item
  \texttt{t(matriz)}: para obtener la transpuesta de la matriz
\item
  \texttt{+}: para sumar matrices
\item
  \texttt{*}: para el producto de un escalar por una matriz
\item
  \texttt{\%*\%}: para multiplicar matrices
\item
  \texttt{mtx.exp(matriz,n)}: para elevar la matriz a \(n\)

  \begin{itemize}
  \tightlist
  \item
    Del paquete \texttt{Biodem}

    \begin{itemize}
    \tightlist
    \item
      No calcula las potencias exactas, las aproxima
    \end{itemize}
  \end{itemize}
\item
  \texttt{\%\^{}\%}: para elevar matrices

  \begin{itemize}
  \tightlist
  \item
    Del paquete \texttt{expm}

    \begin{itemize}
    \tightlist
    \item
      No calcula las potencias exactas, las aproxima
    \end{itemize}
  \end{itemize}
\end{itemize}

\end{frame}

\begin{frame}{Operaciones}
\protect\hypertarget{operaciones-1}{}

\textbf{Ejercicio}

Observad qué ocurre si, siendo
\(A = \begin{pmatrix} 2 & 0 & 2\\ 1 & 2 & 3\\ 0 & 1 & 3 \end{pmatrix}\)
y
\(B = \begin{pmatrix} 3 & 2 & 1\\ 1 & 0 & 0\\ 1 & 1 & 1 \end{pmatrix}\),
realizamos las operaciones \(A*B\), \(A^2\) y \(B^3\)

\includegraphics{Imgs/studytime.png}

\end{frame}

\begin{frame}[fragile]{Operaciones}
\protect\hypertarget{operaciones-2}{}

\begin{itemize}
\tightlist
\item
  \texttt{det(matriz)}: para calcular el determinante de la matriz
\item
  \texttt{qr(matriz)\$rank}: para calcular el rango de la matriz
\item
  \texttt{solve(matriz)}: para calcular la inversa de una matriz
  invertible

  \begin{itemize}
  \tightlist
  \item
    También sirve para resolver sistemas de ecuaciones lineales. Para
    ello introducimos \texttt{solve(matriz,b)}, donde \(b\) es el vector
    de términos independientes
  \end{itemize}
\end{itemize}

\end{frame}

\begin{frame}[fragile]{Valores y vectores propios}
\protect\hypertarget{valores-y-vectores-propios}{}

\href{https://es.wikipedia.org/wiki/Vector_propio_y_valor_propio}{Vector
propio y valor propio}

\begin{itemize}
\tightlist
\item
  \texttt{eigen(matriz)}: para calcular los valores (vaps) y vectores
  propios (veps)

  \begin{itemize}
  \tightlist
  \item
    \texttt{eigen(matriz)\$values}: nos da el vector con los vaps de la
    matriz en orden decreciente de su valor absoluto y repetidos tantas
    veces como su multiplicidad algebraica.
  \item
    \texttt{eigen(matriz)\$vectors}: nos da una matriz cuyas columnas
    son los veps de la matriz.
  \end{itemize}
\end{itemize}

\end{frame}

\begin{frame}[fragile]{Valores y vectores propios}
\protect\hypertarget{valores-y-vectores-propios-1}{}

\begin{Shaded}
\begin{Highlighting}[]
\NormalTok{M =}\StringTok{ }\KeywordTok{rbind}\NormalTok{(}\KeywordTok{c}\NormalTok{(}\DecValTok{2}\NormalTok{,}\DecValTok{6}\NormalTok{,}\OperatorTok{-}\DecValTok{8}\NormalTok{), }\KeywordTok{c}\NormalTok{(}\DecValTok{0}\NormalTok{,}\DecValTok{6}\NormalTok{,}\OperatorTok{-}\DecValTok{3}\NormalTok{), }\KeywordTok{c}\NormalTok{(}\DecValTok{0}\NormalTok{,}\DecValTok{2}\NormalTok{,}\DecValTok{1}\NormalTok{))}
\KeywordTok{eigen}\NormalTok{(M)}
\end{Highlighting}
\end{Shaded}

\begin{verbatim}
eigen() decomposition
$values
[1] 4 3 2

$vectors
          [,1]       [,2] [,3]
[1,] 0.2672612 -0.8164966    1
[2,] 0.8017837  0.4082483    0
[3,] 0.5345225  0.4082483    0
\end{verbatim}

\end{frame}

\begin{frame}{Valores y vectores propios}
\protect\hypertarget{valores-y-vectores-propios-2}{}

\textbf{Ejercicio}

Comprobad, con los datos del ejemplo anterior, que si \(P\) es la matriz
de vectores propios de \(M\) en columna y \(D\) la matriz diagonal cuyas
entradas son los valores propios de \(M\), entoces se cumple la
siguiente igualdad llamada \textbf{descomposición canónica}:
\[M = P\cdot D\cdot P^{-1}\]

\includegraphics{Imgs/studytime.png}

\end{frame}

\begin{frame}[fragile]{Valores y vectores propios}
\protect\hypertarget{valores-y-vectores-propios-3}{}

Si hay algún vap con multiplicidad algebraica mayor que 1 (es decir, que
aparece más de una vez), la función \texttt{eigen()} da tantos valores
de este vap como su multiplicidad algebraica indica. Además, en este
caso, R intenta que los veps asociados a cada uno de estos vaps sean
\href{https://es.wikipedia.org/wiki/Dependencia_e_independencia_lineal}{linealmente
independientes}. Por tanto, cuando como resultado obtenemos veps
repetidos asociados a un vap de multiplicidad algebraica mayor que 1, es
porque para este vap no existen tantos veps linealmente independientes
como su multiplicidad algebraica y, por consiguiente, la matriz no es
\href{https://es.wikipedia.org/wiki/Matriz_diagonalizable}{diagonalizable}.

\end{frame}

\begin{frame}[fragile]{Valores y vectores propios}
\protect\hypertarget{valores-y-vectores-propios-4}{}

\begin{Shaded}
\begin{Highlighting}[]
\NormalTok{M =}\StringTok{ }\KeywordTok{matrix}\NormalTok{(}\KeywordTok{c}\NormalTok{(}\DecValTok{0}\NormalTok{,}\DecValTok{1}\NormalTok{,}\DecValTok{0}\NormalTok{,}\OperatorTok{-}\DecValTok{7}\NormalTok{,}\DecValTok{3}\NormalTok{,}\OperatorTok{-}\DecValTok{1}\NormalTok{,}\DecValTok{16}\NormalTok{,}\OperatorTok{-}\DecValTok{3}\NormalTok{,}\DecValTok{4}\NormalTok{), }\DataTypeTok{nrow=}\DecValTok{3}\NormalTok{, }\DataTypeTok{byrow=}\OtherTok{TRUE}\NormalTok{)}
\KeywordTok{eigen}\NormalTok{(M)}
\end{Highlighting}
\end{Shaded}

\begin{verbatim}
eigen() decomposition
$values
[1] 3 2 2

$vectors
           [,1]       [,2]       [,3]
[1,] -0.1301889 -0.1825742 -0.1825742
[2,] -0.3905667 -0.3651484 -0.3651484
[3,]  0.9113224  0.9128709  0.9128709
\end{verbatim}

\end{frame}

\end{document}
